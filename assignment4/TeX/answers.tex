\documentclass[11pt]{article}
\usepackage[latin1]{inputenc}
\usepackage{a4wide}
\usepackage{amsmath}
\usepackage{amsfonts}
\usepackage{amssymb}
\usepackage{graphicx}
\usepackage{enumerate}
\usepackage{epstopdf}
\usepackage{float}
\usepackage{multicol}
\usepackage{hyperref}
\epstopdfsetup{outdir=./images/}
\usepackage{subcaption}

\renewcommand{\thesubsection}{(\alph{subsection})}
\newcommand{\floor}[1]{\lfloor #1 \rfloor}

\title{Natural Computing, Assignment 3}
\author{Dennis Verheijden - s4455770 \and Pauline Lauron - s1016609 \and Joost Besseling - s4796799}
\begin{document}
\maketitle

\section{}
\subsection{}
\begin{itemize}
	\item 
	The probability that all three doctors give the correct answer is $ 0.8^3 = 0.512$. 
	\item 
	The probability that exactly 2 doctors make the right call is $0.8*0.8*0.2 + 0.8*0.2*0.8 + 0.2*0.8*0.8 = 0.384$. Therefore, the probability that \emph{at least} two doctors make the right call is $0.512 + 0.384 = 0.896$.
	\item  The probability that this group makes the right decision based on majority voting is $0.512 + 0.384 = 0.896$. 
	%(Alternative approach, the probability that they all fail, and that exactly two doctors fail is ...)
\end{itemize}


\subsection{}
The general formula is 
\[
	P(\text{correct predictions} > c/2) \sum_{i = \floor{n/2}}^{n} p^{i}  (1-p)^{n-i}  \binom{n}{i}.
\]

Using this formula, we find a probability of about $0.826$.

\subsection{}
If we use $10000$ runs of the simulations, we get an approximately equal result.

(TODO !!!!)
(DO SOMETHING WITH HOW GOOD THE APPROXIMATION IS?)

\subsection{}
In the comments is a stacklink how to make a nice table

%https://stackoverflow.com/questions/11623056/matplotlib-using-a-colormap-to-color-table-cell-background?utm_medium=organic&utm_source=google_rich_qa&utm_campaign=google_rich_qa








\end{document}