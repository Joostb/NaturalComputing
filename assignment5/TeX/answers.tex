\documentclass[11pt]{article}
\usepackage[latin1]{inputenc}
\usepackage{a4wide}
\usepackage{amsmath}
\usepackage{amsfonts}
\usepackage{amssymb}
\usepackage{graphicx}
\usepackage{enumerate}
\usepackage{epstopdf}
\usepackage{float}
\usepackage{multicol}
\usepackage{hyperref}
\epstopdfsetup{outdir=./images/}
\usepackage{subcaption}

\title{Natural Computing, Assignment 5}
\author{Dennis Verheijden - s4455770 \and Pauline Lauron - s1016609 \and Joost Besseling - s4796799}
\begin{document}
\maketitle

\section{}
a) A nash equilibrium occurs in the states <S,S> and <H,H>. If we are in either of those states, no player can gain anything by changing their strategy, and in fact, they both lose 1.

The candidate ESS's are the NE's, so both <S,S> and <H,H> are candidates. They both are ESS, since $P(S,S) = 1 > 0 = P(H,S)$ and $P(H,H) = 1 > 0 = P(S,H)$.

b) Here, only the state <S,S> is a strict Nash equilibrium. The state <H,H> is a Nash equilibrium, since for both players, the best response to H has 0 payoff, although this payoff is gained by choosing either one of S or H. 

The other states are not Nash equilibria, since the response to S should always be S (for both players, because of the symmetry of the game). <S,S> is also an ESS because $P(S,S) = 1 > 0 = P(H,S)$.  

c) In this game the only Nash equilibrium is <S,S>. The player changing to H will lose 1, so this is not the best response. And there is no ESS because $P(S,S) = 0 <  1 = P(H,S)$ and $P(H,H) = -20$ is the lowest value, so none of the condition is correct in order to be an ESS. 

d) % TODO : add more explanations with the relation between Nash equilibria and ESS.  


%  0,0   -1, 1
%  1,-1, -2,-2


\section{}

a) The ESSs of this game are <A,A> and <B,B> because $P(A,A) = 3 > 0 = P(B,A)$ and $P(B,B) = 1 > 0 = P(A,B)$.

b) A two-player game with strategies A,B and payoffs :  $\pi(A,A) = 3 $, $\pi(A,B) = 0 $, $\pi(B, A) = 0 $, $\pi(B,B) = 1 $ and $x = $ proportion of individuals using A.  
\newline i. The expected payoff are the following : 
\begin{itemize}
	\item $\pi(A,x) = 3x $
	\item $\pi(A,B) = 1-x$
\end{itemize}
ii. $\dot{x} = x(1-x)(\pi(A,x)-\pi(B,x))$
$\dot{x} = x(1-x)(3x-(1-x))$
$\dot{x}= x(1-x)(4x-1)$
\newline iii. Fixed point $x* = 0$, $x* = 1$, $x* = 1/4$
\newline iv.  $x* = 1$, $x* = 0$ are the evolutionary end points because everyone uses strategy A or strategy B respectively. For  $x* = 1/4$, this is not stable. If there is more A, so the A replicate more fast than the B and in the other way, if there is more B, even if A has a bigger paidoff, the B replicate faster than the A.




\end{document}