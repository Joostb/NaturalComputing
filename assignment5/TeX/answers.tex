\documentclass[11pt]{article}
\usepackage[latin1]{inputenc}
\usepackage{a4wide}
\usepackage{amsmath}
\usepackage{amsfonts}
\usepackage{amssymb}
\usepackage{graphicx}
\usepackage{enumerate}
\usepackage{epstopdf}
\usepackage{float}
\usepackage{multicol}
\usepackage{hyperref}
\epstopdfsetup{outdir=./images/}
\usepackage{subcaption}

\title{Natural Computing, Assignment 5}
\author{Dennis Verheijden - s4455770 \and Pauline Lauron - s1016609 \and Joost Besseling - s4796799}
\begin{document}
\maketitle

\section{}
a) A nash equilibrium occurs in the states <S,S> and <H,H>. If we are in either of those states, no player can gain anything by changing their strategy, and in fact, they both lose 1.

The candidate ESS's are the NE's, so both <S,S> and <H,H> are candidates. They both are ESS, since $P(S,S) = 1 > 0 = P(H,S)$ and $P(H,H) = 1 > 0 = P(S,H)$.

b) Here, only the state <S,S> is a strict Nash equilibrium. The state <H,H> is a Nash equilibrium, since for both players, the best response to H has 0 payoff, although this payoff is gained by choosing either one of S or H. 

The other states are not Nash equilibria, since the response to S should always be S (for both players, because of the symmetry of the game).

c) In this game the only Nash equilibrium is <S,S>. The player changing to H will lose 1, so this is not the best response.

d)


%  0,0   -1, 1
%  1,-1, -2,-2

d)


\section{}


\end{document}